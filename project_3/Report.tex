\documentclass[12pt, letterpaper, twoside]{article}
\usepackage[utf8]{inputenc}
\usepackage{graphicx}
\usepackage{amsfonts}
\usepackage{multicol}
\usepackage{hyperref}
\usepackage[margin=2cm,nohead]{geometry}
\usepackage{physics}
\usepackage{xcolor}
\usepackage{amsmath,mathtools}
\usepackage{algorithm}
\usepackage[noend]{algpseudocode}
\usepackage{listings}
\graphicspath{ {images/} }
\begin{document}
\title{FYS-STK 4155 Project 3}
\author{Espen Lønes}
\date{\today}
\maketitle
\ \\
\textbf{Abstract}\\
\ \\
Machine learning regression applied to vinho verde wine samples. The datasets consists of (4898) white and (1599) red wine samples with 11 analytical tests as features. Feature analysis and selection was also done using forward and backward sequential selection. The three models performed adequately, with MSE errors of 0.59, 0.43 (Linear regression - white, red), 0.59, 0.42 (SGD - white, red) and 0.50, 0.47 (Multi layer preceptron - white, red).\\
\section{Introduction}
Portugal is in the top 10 worldwide when it comes to wine production and export. Being responsible for    3.17\% of the worlds wine in 2005 \cite{FAO}. Export of its vinho verde wine has increased greatly in the past years \cite{CVRVV}. In response to this growth the industry has invested in new technologies used in for production and sales purposes. In this context certification and quality assessment is vital. As a part of these processes the wine is often assessed using physiochemical and sensory tests \cite{Ebeler}. The test preformed include measures of i.a. density, alcohol, pH. The sensory tests rely on a human expert. The relation between the sensory and analytical data is complex and not well understood \cite{PvS}. Making wine quality regression/classification a difficult task.\\
\ \\
In this project i attempt to use different ML regression methods to predict the sensory (expert) rating from the analytical measurements like atemplted by P. Cortez et al. \cite{Citation paper}. As mentioned in their paper if classification/regression is possible one could extend this to the consumers. By recommending wines based on their personal sensory preferences.\\
\newpage
\ \\
\section{Theory}
\textbf{Linear regression}\\
\ \\
Linear regression predicts labels for a continuous dependent variable. The prediction is done by calculating the weighted sum of input features. Having inputs $X = (x0, x1, . . . , x{N-1})$ and output y. We get a prediction $\tilde{y}$.
$$
\tilde{y} = \sum_{i=0}^{N-1} X_i \beta_i
$$
where $\beta_j$ are the unknown weights used to train the model in order to minimize the mean squared error. 
\ \\
\textbf{Gradient descent}\\
\ \\
The gradient of a function $f$ for a given variable $x$ tells us how much the function changes with $x$. A function is often depended on multiple variables. It's gradient is then found by finding its partial derivatives.
$$
\nabla f = \left( \frac{\partial f}{\partial x_1}, \frac{\partial f}{\partial x_2}, \cdots \right)
$$
Gradient decent is an iterative method where given a starting point, usually random but may be explicitly chosen. The method moves along the function gradient until it reaches a minimum in the search space (hopefully  the global minimum). For a given task this search space is defined by chosen cost function, quantifying how far from the optimal/best solution our model is. Since our goal is to reach the minimum we move in the opposite direction of the gradient. This direction is found by calculating the gradient of the search space at the current point. We then move a given length in that direction to get our new solution. This repeats until we get to a minimum where the gradient is zero. If we are at point $w_i$ with cost function $C(w)$ we get the iterative scheme.
$$
w_{i+1} = w_i - \eta_k \nabla_w C(w_i)
$$
Where $\eta_k$ is the step length commonly names learning rate. The point $w$ corresponds to a set of values we call weights.\\
\ \\
As previously stated our goal is to reach the global minimum. But depending on cost function choice, leaning rate, initial guess and problem complexity we may (and usually will) end up in a local minimum instead. A dilemma with learning rate is that having to small a learning rate makes the training slow as we need many iterations until we reach a minimum. On the other hand if it is to large we will probably overshoot the minimum. A popular method for mitigating this is by having a decreasing learning rate. With the idea that we move fast towards a minimum, as we get close the learning rate decreases making it less likely we overshoot the minimum.\\
\newpage
\ \\
\textbf{Stochastic gradient descent}\\
\ \\
Calculating the cost function gradient can be computationally heavy. We may instead use a version of gradient descent called stochastic gradient descent to speedup this process. SGD works by dividing the points into groups called mini-batches. We then choose a random batch and compute its gradient. If the sample size is large enough the average of these gradients will approximate the gradient calculated for all data points. The flow of the SGD method is therefore as follows. The SDG picks a random mini-batch of training points and calculates its gradient. Then the next mini-batch is chosen until all batches are used. This is called one epoch. We then use these as the gradient in gradient descent and may also then start another epoch. Since this method only approximates the gradients means that some randomness is introduced in the gradient. This has the added benefit of possibly getting us out of a local minima.\\
\ \\
As a step to further increase the algorithms speed, SGD is often used with a momentum factor. The momentum increases the acceleration of the gradient giving a higher convergence rate. Defining momentum $0 \leq \gamma \leq 1$ we get this scheme for our weights.
$$
v_{i+1} = \gamma v_i + \eta_k \nabla C(w_i)
$$
$$
w_{i+1} = w_i - v_{i+1}
$$
(Usualy having $v_{-1} = 0$)\\
High $\gamma$ values gives large acceleration. With $\gamma = 0$ being ordinary SGD.\\
\ \\
\ \\
\textbf{Neural networks}\\
\ \\
A neural network consists of layers of nodes, one input, one output and a any number of hidden layers (may have no hidden layers).
Each layer has a set of neurons/nodes, for the input layer these are the input neurons where the number of neurons are determined by our problem and our representation of it (the features). The output neurons represent our solution/answer.\\
\includegraphics[scale=0.5]{"NN"}\\
\newpage
\ \\
\textbf{Feed forward neural network (FFNN)}\\ 
\ \\
A type of neural network where the output from one layer is used as input to the next. These are the simplest and easiest forms of neural network and use algorithms like back propagation to adjust the neuron weights.\\
\ \\
\ \\
\textbf{Activation functions}\\
\ \\
In neural networks the sum of the weighted inputs to a node, is used as input to that nodes activation function. Activation functions are divided in two main groups, linear and non-linear. The linear functions are lines/hyperplanes and are not bounded by any range, the identity function is such a function $\sigma(z) = z$.\\
The non-linear functions are more often used as they usually make the model better at generalizing and adapting to different data. (for non-linearly separable problems, which most are).\\
\ \\
Sigmoid functions are a set of s-shaped functions often used as activation functions in neural networks. One specific sigmoid that has been used frequently in NNs is the logistic function, given by.
$$
\sigma(z) = \frac{1}{1 + e^{-z}} = \frac{e^z}{e^z + 1}
$$
With the derivative
$$
\sigma'(z) = \sigma(z)(1 - \sigma(z))
$$
\includegraphics[scale=1]{"log_func"}\\
Plot of the logistic function.\\
\ \\
Its advantage lies in that it's values fall between 0 and 1. This is very useful when the output is a probability. Unfortunately the logistic function can halt the NN as the derivative is often very close to zero. We now often use functions like ReLU and leaky ReLU instead.\\
\newpage
\ \\
The ReLU (rectified linear unit) function is variant of the sigmoid.\\
\[
    \sigma(z) =
    \begin{dcases*}
  		\; 0 & if $z \leq 0 $ \\
  		z & if $z > 0$
	\end{dcases*}
	= max\{0, z\}
\]
Derivative\\
\[
    \sigma'(z) =
    \begin{dcases*}
  		\; 0 & if $z \leq 0 $ \\
  		1 & if $z > 0$
	\end{dcases*}
\]
We see that for ReLU increasing the weights does not saturate the learning. But it does if the weights are negative. ReLU is often used for the hidden layers.\\
\ \\
Another variation is the Leaky/Parametric ReLU that does not saturate for negative weights, instead of zero for negative input it multiples it with a (usually quite small e.g. 0.01).\\
\[
    \sigma(z) =
    \begin{dcases*}
  		\; az & if $z \leq 0 $ \\
  		z & if $z > 0$
	\end{dcases*}
\]
\[
    \sigma'(z) =
    \begin{dcases*}
  		\; a & if $z \leq 0 $ \\
  		1 & if $z > 0$
	\end{dcases*}
\]
As the gradient of leaky ReLU newer goes to zero the neurons never stop learning. Making this activation function very robust.\\
\includegraphics[scale=1]{"ReLU"}\\
ReLU on the left. Parametric ReLU on the right.\\
\ \\
\ \\
\textbf{Cost functions}\\
\ \\
Cost functions estimates how right or wrong a prediction is, by a given metric. It uses predicted values and a true values in order to quantify the quality of our current model. 
In neural networks this value is used to update the weights and biases. For liner regression we use the mean squared error (MSE), while for logistic regression cross entropy is used.\\
\ \\
The MSE cost function.
$$
C(w) = (Xw - z)^T(Xw - z) + \lambda w^T w
$$
Where $X$ is the design/feature matrix, $z$ is the true labels, w is the weights and $\lambda$ is the regularization parameter.\\
\ \\
MSE takes the averages of the squares of the errors. This works well when fitting to a line. For gradient descent we also needs its derivative.
$$
\nabla C(w) = 2 X^T (X^T w - z) + 2 \lambda w
$$
\ \\
\ \\
The R2 score is the proportion of the variation of the independent variable, that is predicted from the independent variables.\\
If $\tilde{y}$ is the mean of the data. And:
$$
SS_{res} = sum_i (y_i - f_i)^2 
$$
$$
SS_{tot} = sum_i (y_i - \tilde{y})^2
$$
Then.
$$
R^2 = 1 - \frac{SS_{res}}{SS_{tot}}
$$
\ \\
\textbf{Backpropagation}\\
\ \\
Backpropagation is an algorithm used in updating the weights and biases of a neural network. It computes the gradient of the cost function in respect to the current weights and biases. The aim of backpropagation is to calculate the partial derivatives.
$\frac{\partial C}{\partial w}$ and $\frac{\partial C}{\partial b}$\\
This is done using the chain rule. Where the cost function depends on the activation function $a$, which in turn depends on the weighted sum $z$, which again depends on the weights and biases. Giving us.\\
$$
\frac{\partial C}{\partial w} = \frac{\partial C}{\partial a}\frac{\partial C}{\partial z}\frac{\partial C}{\partial w}
$$
\ \\
$$
\frac{\partial C}{\partial b} = \frac{\partial C}{\partial a}\frac{\partial C}{\partial z}\frac{\partial C}{\partial b}
$$
Doing this individually for all weights and biases is very slow. Instead we calculate the gradient for each layer in reverse order. By doing this we avoid unnecessary calculations and duplicate values. It also makes it easier to see how changing the weights and biases change the output when starting on the last layer. The gradient of the weighted input(s) to a given hidden layer is called $\delta^l$, (often called the error). We also use k to denote the k-th node in the $(l-1)$-th layer. And $j$ for the $j$-th node in the $l$-t layer.\\
\ \\
Backpropagation consists of a set of four equations. The first being for the error $\delta^l$ in the output layer.\\
$$
\delta_k^L = \frac{\partial C}{\partial a_j^L}\sigma'(z_j^L)
$$
Thereafter the error for the hidden layers $L-1, L-2,\cdots,2$ are computed recursively.\\
$$
\delta_j^l = \sum_k \delta_j^{l+1} w_j^{l+1} \sigma'(z_j^{l})
$$
These errors are then used to calculate the cost functions partial derivatives.\\
$$
\frac{\partial C}{\partial w_{jk}^{l}} = \delta_j^l a_k^{l-1}
$$
\ \\
$$
\frac{\partial C}{\partial b_j^l} = \delta_j^l
$$
Lastly the partial derivatives are used to update the weights and biases for layers $L-1,L-2,\cdots,2$ using gradient decent.\\
$$
w_{jk}^l \leftarrow w_{jk}^l - \eta \frac{\partial C}{\partial w_{jk}^l}
= w_{jk}^l - \eta \delta_j^l a_{k}^{l-1}
$$
\ \\
$$
b_j^l \leftarrow b_j^l - \eta \frac{\partial C}{\partial b_j^l}
= b_j^l - \eta \delta_j^l
$$
In order to avoid overfitting a regularization parameter $\lambda$ may be added. This penalises the weight matrices. A common regularization parameter is the L2. The penalty is added to the cost function. The regularization forces the weights towards zero but not all the way to zero. This way a simpler and more generalizable solution with the same error will likely be selected. Thus reducing overfitting. The cost function with L2 is defined by.
$$
C(\theta) = L(\theta) + \lambda||w||_2^2
$$
Having gradient:
$$
\nabla C(\theta) = \nabla L(\theta) + 2 \lambda w
$$
\newpage
\ \\
\textbf{Confusion matrix}\\
\ \\
The Confusion matrix is a simple yet extremely useful analysis tool when doing machine learning.\\
It works for binary and multi-class problems by plotting how many of a given class was classified as the different classes.\\
\ \\
\includegraphics[scale=0.75]{"CM.jpg"}\\
\ \\
Here we also see three other metrics, sensitivity, specificity and accuracy. Which each provide a different look on how good our model is for different types of error.\\
\section{Method}
\textbf{Feature Selection}\\
\ \\
I used sequential feature selection for the feature selection. This is done by (for the forward version) starting with no features and selecting the one giving the best cross-validation score. This is repeated until the desired amount of features is reached. For the backwards method we start with all features and remove one at a time instead. Both forward and backward was used to do feature selection. Sckitlearns SequentialFeatureSelector method was used\\
\ \\
\ \\
\textbf{ML models}\\
\ \\
After feature selection i trained tree different models on the data. Linear regression, SGD regression and multi layer preseptron (MLP) regression. All models used their sckitlearns versions/implementations. The models hyperparameters where tweaked using the training data.\\
\ \\
\ \\

\section{Results}
Correlation matrices for white wine (left) and red wine (right).\\
\begin{multicols}{2}
	\includegraphics[scale=0.65]{"corr_white"}\\
	\includegraphics[scale=0.65]{"corr_red"}\\
\end{multicols}
We see that both red and white wine has about 0.6 in correlation between free and total sulphur dioxide. Not so surprising as the free is probably a big part of the total.\\
The rest of the stronger correlations are different between the two wine types.\\
For white wine there is about 0.7 in correlation between density-alcohol and density-residual sugar.
Also not surprising as sugar is denser than water and alcohol is less dense than water. But still a little surprising is that the correlation is not noticeable in the red wine.\\
For the red wine, the largest correlations are between, fixed acidity-pH, fixed acidity-density and fixed acidity- citric acid.\\
We see from the correlation matrices that we probably need many of the features for the ML model to be trained sufficiently, as many of the features have low correlation. But some of the higher correlated combinations may be redundant. We also see from the difference that the two wine types may need a different feature combinations.\\
\newpage
\ \\
Target/label distributions:\\
\begin{multicols}{2}
	\includegraphics[scale=0.6]{"histogram_white"}\\
	\includegraphics[scale=0.6]{"histogram_red"}\\
\end{multicols}
There is a large difference in the label distribution. With a low (or none at all) amount of samples for both very good and very bad wine. This will make it difficult to train the model on the types of wine with low sample numbers. Wines with 1,2,9 or 10 as score if collected and used later to test the model will probably be miss labelled as such wins where not in the training sample.\\
\ \\
\ \\
Minimum, maximum and mean of the different features.\\
\begin{verbatim}
Red wine
                          min      max       mean
fixed acidity             4.6     15.9    8.31964
volatile acidity         0.12     1.58   0.527821
citric acid                 0        1   0.270976
residual sugar            0.9     15.5    2.53881
chlorides               0.012    0.611  0.0874665
free sulfur dioxide         1       72    15.8749
total sulfur dioxide        6      289    46.4678
density               0.99007  1.00369   0.996747
pH                       2.74     4.01    3.31111
sulphates                0.33        2   0.658149
alcohol                   8.4     14.9     10.423
\end{verbatim}
\newpage
\ \\
\begin{verbatim}
White wine
                          min      max       mean
fixed acidity             3.8     14.2    6.85479
volatile acidity         0.08      1.1   0.278241
citric acid                 0     1.66   0.334192
residual sugar            0.6     65.8    6.39141
chlorides               0.009    0.346  0.0457724
free sulfur dioxide         2      289    35.3081
total sulfur dioxide        9      440    138.361
density               0.98711  1.03898   0.994027
pH                       2.72     3.82    3.18827
sulphates                0.22     1.08   0.489847
alcohol                     8     14.2    10.5143
\end{verbatim}
From this we see there is a large scale difference between the different features. Scaling the data is then probably a good idea to avoid some features being overrepresented because of their scale. I chose to use sklearn standard scaler.\\
\newpage
\ \\
The forward and backward sequential feature selection resulted in the flowing plots.\\
For white wine:\\
\includegraphics[scale=0.75]{"forward_v_backward_white"}\\
For red wine:\\
\includegraphics[scale=0.75]{"forward_v_backward_red"}\\
\ \\
We see that for this type of feature selection it finds that the combination of all features is optimal.\\
Using this in the different models I get these results.\\
\ \\
\begin{multicols}{2}
White wine:\\
\begin{verbatim}
LR:
MSE: 0.5889700125908756
R2: 0.24824729577620164
MAD: 0.5964982518058798
SGD:
MSE: 0.588102748404873
R2: 0.24935425909040476
MAD: 0.5955205859534801
MLP:
MSE: 0.5017647241025605
R2: 0.35955484971315554
MAD: 0.5482520202440326
\end{verbatim}
\ \\
Red wine:
\begin{verbatim}
LR:
MSE: 0.4267989097992499
R2: 0.3653383966842185
MAD: 0.506160827121204
SGD:
MSE: 0.4247741608117259
R2: 0.3683492535754378
MAD: 0.505201285358217
MLP:
MSE: 0.4764476462328574
R2: 0.2915093733571037
MAD: 0.4914478390771567
\end{verbatim}
\end{multicols}
\section{Conclusion/Discussion}
There where some differences between the different methods, but not a lot. The MLP had the best results for white wine. While for the red wine MLP was worst, with LR and SGD being equally good. This is surprising as even though none of the models had extremely impressive results. I would have expected the best model for both sets to be the same. It is apparent that more work need to be done to test whether or not the two datasets need different models. In addition further hyper parameter adjustment or testing of more models to achieve higher scores.\\
\newpage
\ \\ 
\begin{thebibliography}{9}
\bibitem{UCI}
\url{https://archive.ics.uci.edu/ml/datasets/Wine+Quality}

\bibitem{Citation paper}
P. Cortez, A. Cerdeira, F. Almeida, T. Matos and J. Reis. Modeling wine preferences by data mining from physicochemical properties. In Decision Support Systems, Elsevier, 47(4):547-553, 2009. (\url{https://www.sciencedirect.com/science/article/pii/S0167923609001377?via%3Dihub})

\bibitem{Cortez}
Paulo Cortez, University of Minho, Guimarães, Portugal, \url{http://www3.dsi.uminho.pt/pcortez}
A. Cerdeira, F. Almeida, T. Matos and J. Reis, Viticulture Commission of the Vinho Verde Region(CVRVV), Porto, Portugal
@2009

\bibitem{Vino Verde}
\url{https://www.vinhoverde.pt/en/}

\bibitem{FAO}
FAO\\
FAOSTAT — Food and Agriculture Organization Agriculture Trade Domain Statistics (July 2008)\\
\url{http://faostat.fao.org/site/535/DesktopDefault.aspx?PageID=535}\\

\bibitem{CVRVV}
CVRVV. Portuguese Wine — Vinho Verde.\\
Comissão de Viticultura da Região dos Vinhos Verdes (CVRVV),\\ 
\url{http://www.vinhoverde.pt}, July 2008.\\

\bibitem{Ebeler}
S. Ebeler
Flavor Chemistry — Thirty Years of Progress, Kluwer Academic Publishers (1999), pp. 409-422
\url{https://link.springer.com/chapter/10.1007%2F978-1-4615-4693-1_35}

\bibitem{PvS}
A. Legin, A. Rudnitskaya, L. Luvova, Y. Vlasov, C. Natale, A. D'Amico
Evaluation of Italian wine by the electronic tongue: recognition, quantitative analysis and correlation with human sensory perception
Analytica Chimica Acta, 484 (1) (2003), pp. 33-34
\url{https://www.scopus.com/record/display.uri?eid=2-s2.0-0038066312&origin=inward&txGid=cdf7d5eb8afe889f892d9b354a7e1317}

\bibitem{p2}
\url{https://github.com/Espen-git/FYS-STK4155/blob/master/project_2/Report.pdf}
\end{thebibliography}
\end{document}